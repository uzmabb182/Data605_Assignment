% Options for packages loaded elsewhere
\PassOptionsToPackage{unicode}{hyperref}
\PassOptionsToPackage{hyphens}{url}
%
\documentclass[
]{article}
\usepackage{amsmath,amssymb}
\usepackage{iftex}
\ifPDFTeX
  \usepackage[T1]{fontenc}
  \usepackage[utf8]{inputenc}
  \usepackage{textcomp} % provide euro and other symbols
\else % if luatex or xetex
  \usepackage{unicode-math} % this also loads fontspec
  \defaultfontfeatures{Scale=MatchLowercase}
  \defaultfontfeatures[\rmfamily]{Ligatures=TeX,Scale=1}
\fi
\usepackage{lmodern}
\ifPDFTeX\else
  % xetex/luatex font selection
\fi
% Use upquote if available, for straight quotes in verbatim environments
\IfFileExists{upquote.sty}{\usepackage{upquote}}{}
\IfFileExists{microtype.sty}{% use microtype if available
  \usepackage[]{microtype}
  \UseMicrotypeSet[protrusion]{basicmath} % disable protrusion for tt fonts
}{}
\makeatletter
\@ifundefined{KOMAClassName}{% if non-KOMA class
  \IfFileExists{parskip.sty}{%
    \usepackage{parskip}
  }{% else
    \setlength{\parindent}{0pt}
    \setlength{\parskip}{6pt plus 2pt minus 1pt}}
}{% if KOMA class
  \KOMAoptions{parskip=half}}
\makeatother
\usepackage{xcolor}
\usepackage[margin=1in]{geometry}
\usepackage{color}
\usepackage{fancyvrb}
\newcommand{\VerbBar}{|}
\newcommand{\VERB}{\Verb[commandchars=\\\{\}]}
\DefineVerbatimEnvironment{Highlighting}{Verbatim}{commandchars=\\\{\}}
% Add ',fontsize=\small' for more characters per line
\usepackage{framed}
\definecolor{shadecolor}{RGB}{248,248,248}
\newenvironment{Shaded}{\begin{snugshade}}{\end{snugshade}}
\newcommand{\AlertTok}[1]{\textcolor[rgb]{0.94,0.16,0.16}{#1}}
\newcommand{\AnnotationTok}[1]{\textcolor[rgb]{0.56,0.35,0.01}{\textbf{\textit{#1}}}}
\newcommand{\AttributeTok}[1]{\textcolor[rgb]{0.13,0.29,0.53}{#1}}
\newcommand{\BaseNTok}[1]{\textcolor[rgb]{0.00,0.00,0.81}{#1}}
\newcommand{\BuiltInTok}[1]{#1}
\newcommand{\CharTok}[1]{\textcolor[rgb]{0.31,0.60,0.02}{#1}}
\newcommand{\CommentTok}[1]{\textcolor[rgb]{0.56,0.35,0.01}{\textit{#1}}}
\newcommand{\CommentVarTok}[1]{\textcolor[rgb]{0.56,0.35,0.01}{\textbf{\textit{#1}}}}
\newcommand{\ConstantTok}[1]{\textcolor[rgb]{0.56,0.35,0.01}{#1}}
\newcommand{\ControlFlowTok}[1]{\textcolor[rgb]{0.13,0.29,0.53}{\textbf{#1}}}
\newcommand{\DataTypeTok}[1]{\textcolor[rgb]{0.13,0.29,0.53}{#1}}
\newcommand{\DecValTok}[1]{\textcolor[rgb]{0.00,0.00,0.81}{#1}}
\newcommand{\DocumentationTok}[1]{\textcolor[rgb]{0.56,0.35,0.01}{\textbf{\textit{#1}}}}
\newcommand{\ErrorTok}[1]{\textcolor[rgb]{0.64,0.00,0.00}{\textbf{#1}}}
\newcommand{\ExtensionTok}[1]{#1}
\newcommand{\FloatTok}[1]{\textcolor[rgb]{0.00,0.00,0.81}{#1}}
\newcommand{\FunctionTok}[1]{\textcolor[rgb]{0.13,0.29,0.53}{\textbf{#1}}}
\newcommand{\ImportTok}[1]{#1}
\newcommand{\InformationTok}[1]{\textcolor[rgb]{0.56,0.35,0.01}{\textbf{\textit{#1}}}}
\newcommand{\KeywordTok}[1]{\textcolor[rgb]{0.13,0.29,0.53}{\textbf{#1}}}
\newcommand{\NormalTok}[1]{#1}
\newcommand{\OperatorTok}[1]{\textcolor[rgb]{0.81,0.36,0.00}{\textbf{#1}}}
\newcommand{\OtherTok}[1]{\textcolor[rgb]{0.56,0.35,0.01}{#1}}
\newcommand{\PreprocessorTok}[1]{\textcolor[rgb]{0.56,0.35,0.01}{\textit{#1}}}
\newcommand{\RegionMarkerTok}[1]{#1}
\newcommand{\SpecialCharTok}[1]{\textcolor[rgb]{0.81,0.36,0.00}{\textbf{#1}}}
\newcommand{\SpecialStringTok}[1]{\textcolor[rgb]{0.31,0.60,0.02}{#1}}
\newcommand{\StringTok}[1]{\textcolor[rgb]{0.31,0.60,0.02}{#1}}
\newcommand{\VariableTok}[1]{\textcolor[rgb]{0.00,0.00,0.00}{#1}}
\newcommand{\VerbatimStringTok}[1]{\textcolor[rgb]{0.31,0.60,0.02}{#1}}
\newcommand{\WarningTok}[1]{\textcolor[rgb]{0.56,0.35,0.01}{\textbf{\textit{#1}}}}
\usepackage{graphicx}
\makeatletter
\def\maxwidth{\ifdim\Gin@nat@width>\linewidth\linewidth\else\Gin@nat@width\fi}
\def\maxheight{\ifdim\Gin@nat@height>\textheight\textheight\else\Gin@nat@height\fi}
\makeatother
% Scale images if necessary, so that they will not overflow the page
% margins by default, and it is still possible to overwrite the defaults
% using explicit options in \includegraphics[width, height, ...]{}
\setkeys{Gin}{width=\maxwidth,height=\maxheight,keepaspectratio}
% Set default figure placement to htbp
\makeatletter
\def\fps@figure{htbp}
\makeatother
\setlength{\emergencystretch}{3em} % prevent overfull lines
\providecommand{\tightlist}{%
  \setlength{\itemsep}{0pt}\setlength{\parskip}{0pt}}
\setcounter{secnumdepth}{-\maxdimen} % remove section numbering
\ifLuaTeX
  \usepackage{selnolig}  % disable illegal ligatures
\fi
\IfFileExists{bookmark.sty}{\usepackage{bookmark}}{\usepackage{hyperref}}
\IfFileExists{xurl.sty}{\usepackage{xurl}}{} % add URL line breaks if available
\urlstyle{same}
\hypersetup{
  pdftitle={Data 605 HW2},
  pdfauthor={Mubashira Qari},
  hidelinks,
  pdfcreator={LaTeX via pandoc}}

\title{Data 605 HW2}
\author{Mubashira Qari}
\date{2024-01-29}

\begin{document}
\maketitle

\hypertarget{problem-set-1}{%
\subsection{PROBLEM SET 1:}\label{problem-set-1}}

\hypertarget{show-that-ata-aat-in-general.-proof-and-demonstration.}{%
\subsubsection{(1) Show that ATA != AAT in general. (Proof and
demonstration.)}\label{show-that-ata-aat-in-general.-proof-and-demonstration.}}

\hypertarget{load-the-packages}{%
\subsubsection{Load the Packages}\label{load-the-packages}}

\begin{Shaded}
\begin{Highlighting}[]
\FunctionTok{library}\NormalTok{(matrixcalc)}
\FunctionTok{library}\NormalTok{(matlib)}
\end{Highlighting}
\end{Shaded}

\begin{verbatim}
## 
## Attaching package: 'matlib'
\end{verbatim}

\begin{verbatim}
## The following object is masked from 'package:matrixcalc':
## 
##     vec
\end{verbatim}

\hypertarget{creating-a-3-x-3-matrix}{%
\subsubsection{Creating a 3 X 3 Matrix}\label{creating-a-3-x-3-matrix}}

\begin{Shaded}
\begin{Highlighting}[]
\NormalTok{A }\OtherTok{=} \FunctionTok{matrix}\NormalTok{(}\FunctionTok{c}\NormalTok{(}\DecValTok{1}\NormalTok{, }\DecValTok{1}\NormalTok{, }\DecValTok{2}\NormalTok{, }\DecValTok{3}\NormalTok{, }\DecValTok{2}\NormalTok{, }\DecValTok{2}\NormalTok{, }\DecValTok{1}\NormalTok{, }\DecValTok{1}\NormalTok{, }\DecValTok{1}\NormalTok{), }\AttributeTok{nrow =} \DecValTok{3}\NormalTok{)}
\NormalTok{A}
\end{Highlighting}
\end{Shaded}

\begin{verbatim}
##      [,1] [,2] [,3]
## [1,]    1    3    1
## [2,]    1    2    1
## [3,]    2    2    1
\end{verbatim}

\hypertarget{finding-the-determinant-of-a-to-check-if-deta-0-then-inverse-exists}{%
\subsubsection{Finding the determinant of A to check: if det(A) != 0
then inverse
exists}\label{finding-the-determinant-of-a-to-check-if-deta-0-then-inverse-exists}}

\begin{Shaded}
\begin{Highlighting}[]
\FunctionTok{det}\NormalTok{(A)}
\end{Highlighting}
\end{Shaded}

\begin{verbatim}
## [1] 1
\end{verbatim}

\hypertarget{finding-the-tranpose-of-a}{%
\subsubsection{Finding the tranpose of
A}\label{finding-the-tranpose-of-a}}

\begin{Shaded}
\begin{Highlighting}[]
\NormalTok{AT }\OtherTok{\textless{}{-}} \FunctionTok{t}\NormalTok{(A)}

\NormalTok{AT}
\end{Highlighting}
\end{Shaded}

\begin{verbatim}
##      [,1] [,2] [,3]
## [1,]    1    1    2
## [2,]    3    2    2
## [3,]    1    1    1
\end{verbatim}

\hypertarget{mutiplying-a-ai}{%
\subsubsection{Mutiplying A AI}\label{mutiplying-a-ai}}

\begin{Shaded}
\begin{Highlighting}[]
\NormalTok{A }\SpecialCharTok{\%*\%}\NormalTok{ AT}
\end{Highlighting}
\end{Shaded}

\begin{verbatim}
##      [,1] [,2] [,3]
## [1,]   11    8    9
## [2,]    8    6    7
## [3,]    9    7    9
\end{verbatim}

\hypertarget{mutiplying-ai-a}{%
\subsubsection{Mutiplying AI A}\label{mutiplying-ai-a}}

\begin{Shaded}
\begin{Highlighting}[]
\NormalTok{AT }\SpecialCharTok{\%*\%}\NormalTok{ A}
\end{Highlighting}
\end{Shaded}

\begin{verbatim}
##      [,1] [,2] [,3]
## [1,]    6    9    4
## [2,]    9   17    7
## [3,]    4    7    3
\end{verbatim}

\hypertarget{checking-for-proof-and-it-proves-that-a-x-ai-ai-x-a}{%
\subsubsection{Checking for proof and it proves that (A X AI != AI X
A)}\label{checking-for-proof-and-it-proves-that-a-x-ai-ai-x-a}}

\begin{Shaded}
\begin{Highlighting}[]
\NormalTok{(A }\SpecialCharTok{\%*\%}\NormalTok{ AT }\SpecialCharTok{==}\NormalTok{ AT }\SpecialCharTok{\%*\%}\NormalTok{ A)}
\end{Highlighting}
\end{Shaded}

\begin{verbatim}
##       [,1]  [,2]  [,3]
## [1,] FALSE FALSE FALSE
## [2,] FALSE FALSE  TRUE
## [3,] FALSE  TRUE FALSE
\end{verbatim}

\hypertarget{for-a-special-type-of-square-matrix-a-we-get-at-a-aat.-under-what-conditionscould-this-be-true-hint-the-identity-matrix-i-is-an-example-of-such-a-matrix.}{%
\subsubsection{(2) For a special type of square matrix A, we get AT A =
AAT. Under what conditionscould this be true? (Hint: The Identity matrix
I is an example of such a
matrix).}\label{for-a-special-type-of-square-matrix-a-we-get-at-a-aat.-under-what-conditionscould-this-be-true-hint-the-identity-matrix-i-is-an-example-of-such-a-matrix.}}

\hypertarget{creating-an-identity-matrix-for-checking-the-case-of-a-special-square-matrix}{%
\subsubsection{Creating an identity matrix for checking the case of a
special square
matrix}\label{creating-an-identity-matrix-for-checking-the-case-of-a-special-square-matrix}}

\begin{Shaded}
\begin{Highlighting}[]
\NormalTok{I }\OtherTok{\textless{}{-}} \FunctionTok{diag}\NormalTok{(}\DecValTok{3}\NormalTok{)}

\NormalTok{I}
\end{Highlighting}
\end{Shaded}

\begin{verbatim}
##      [,1] [,2] [,3]
## [1,]    1    0    0
## [2,]    0    1    0
## [3,]    0    0    1
\end{verbatim}

\hypertarget{cretaing-transpose-of-a}{%
\subsubsection{Cretaing transpose of A}\label{cretaing-transpose-of-a}}

\begin{Shaded}
\begin{Highlighting}[]
\NormalTok{A }\OtherTok{\textless{}{-}}\NormalTok{ I}
\NormalTok{AT }\OtherTok{\textless{}{-}} \FunctionTok{t}\NormalTok{(A)}
\NormalTok{AT}
\end{Highlighting}
\end{Shaded}

\begin{verbatim}
##      [,1] [,2] [,3]
## [1,]    1    0    0
## [2,]    0    1    0
## [3,]    0    0    1
\end{verbatim}

\begin{Shaded}
\begin{Highlighting}[]
\NormalTok{AT }\SpecialCharTok{\%*\%}\NormalTok{ A }
\end{Highlighting}
\end{Shaded}

\begin{verbatim}
##      [,1] [,2] [,3]
## [1,]    1    0    0
## [2,]    0    1    0
## [3,]    0    0    1
\end{verbatim}

\begin{Shaded}
\begin{Highlighting}[]
\NormalTok{A }\SpecialCharTok{\%*\%}\NormalTok{ AT}
\end{Highlighting}
\end{Shaded}

\begin{verbatim}
##      [,1] [,2] [,3]
## [1,]    1    0    0
## [2,]    0    1    0
## [3,]    0    0    1
\end{verbatim}

For a special type of square matrix called Identity Matrix, we get AT A
= A AT

\begin{Shaded}
\begin{Highlighting}[]
\NormalTok{(A }\SpecialCharTok{\%*\%}\NormalTok{ AT }\SpecialCharTok{==}\NormalTok{ AT }\SpecialCharTok{\%*\%}\NormalTok{ A)}
\end{Highlighting}
\end{Shaded}

\begin{verbatim}
##      [,1] [,2] [,3]
## [1,] TRUE TRUE TRUE
## [2,] TRUE TRUE TRUE
## [3,] TRUE TRUE TRUE
\end{verbatim}

\hypertarget{problem-set-2}{%
\subsection{PROBLEM SET 2:}\label{problem-set-2}}

\begin{Shaded}
\begin{Highlighting}[]
\NormalTok{A }\OtherTok{=} \FunctionTok{matrix}\NormalTok{(}\FunctionTok{c}\NormalTok{(}\DecValTok{1}\NormalTok{, }\DecValTok{4}\NormalTok{, }\DecValTok{3}\NormalTok{, }\DecValTok{1}\NormalTok{,}\DecValTok{3}\NormalTok{, }\DecValTok{5}\NormalTok{, }\DecValTok{1}\NormalTok{, }\SpecialCharTok{{-}}\DecValTok{1}\NormalTok{, }\DecValTok{3}\NormalTok{), }\AttributeTok{nrow =} \DecValTok{3}\NormalTok{)}
\NormalTok{A}
\end{Highlighting}
\end{Shaded}

\begin{verbatim}
##      [,1] [,2] [,3]
## [1,]    1    1    1
## [2,]    4    3   -1
## [3,]    3    5    3
\end{verbatim}

\hypertarget{in-order-to-eliminate-the-2-in-the-2nd-row-we-can-set-up-the-following-elimination-matrix-e21}{%
\subsubsection{In order to eliminate the 2 in the 2nd row, we can set up
the following elimination matrix
E21:}\label{in-order-to-eliminate-the-2-in-the-2nd-row-we-can-set-up-the-following-elimination-matrix-e21}}

\begin{Shaded}
\begin{Highlighting}[]
\NormalTok{E21 }\OtherTok{=} \FunctionTok{matrix}\NormalTok{(}\FunctionTok{c}\NormalTok{(}\DecValTok{1}\NormalTok{,}\SpecialCharTok{{-}}\DecValTok{4}\NormalTok{,}\DecValTok{0}\NormalTok{,}\DecValTok{0}\NormalTok{,}\DecValTok{1}\NormalTok{,}\DecValTok{0}\NormalTok{,}\DecValTok{0}\NormalTok{,}\DecValTok{0}\NormalTok{,}\DecValTok{1}\NormalTok{),}\AttributeTok{nrow=}\DecValTok{3}\NormalTok{)}
\NormalTok{E21}
\end{Highlighting}
\end{Shaded}

\begin{verbatim}
##      [,1] [,2] [,3]
## [1,]    1    0    0
## [2,]   -4    1    0
## [3,]    0    0    1
\end{verbatim}

\begin{Shaded}
\begin{Highlighting}[]
\NormalTok{E21 }\SpecialCharTok{\%*\%}\NormalTok{ A}
\end{Highlighting}
\end{Shaded}

\begin{verbatim}
##      [,1] [,2] [,3]
## [1,]    1    1    1
## [2,]    0   -1   -5
## [3,]    3    5    3
\end{verbatim}

\hypertarget{to-eliminate-3-in-row-3-we-can-choose-e32-to-be}{%
\subsubsection{To eliminate 3 in row 3, we can choose E32 to
be:}\label{to-eliminate-3-in-row-3-we-can-choose-e32-to-be}}

\begin{Shaded}
\begin{Highlighting}[]
\NormalTok{E31 }\OtherTok{=} \FunctionTok{matrix}\NormalTok{(}\FunctionTok{c}\NormalTok{(}\DecValTok{1}\NormalTok{,}\DecValTok{0}\NormalTok{,}\SpecialCharTok{{-}}\DecValTok{3}\NormalTok{,}\DecValTok{0}\NormalTok{,}\DecValTok{1}\NormalTok{,}\DecValTok{0}\NormalTok{,}\DecValTok{0}\NormalTok{,}\DecValTok{0}\NormalTok{,}\DecValTok{1}\NormalTok{),}\AttributeTok{nrow=}\DecValTok{3}\NormalTok{)}
\NormalTok{E31}
\end{Highlighting}
\end{Shaded}

\begin{verbatim}
##      [,1] [,2] [,3]
## [1,]    1    0    0
## [2,]    0    1    0
## [3,]   -3    0    1
\end{verbatim}

\begin{Shaded}
\begin{Highlighting}[]
\NormalTok{E31 }\SpecialCharTok{\%*\%}\NormalTok{ E21 }\SpecialCharTok{\%*\%}\NormalTok{ A}
\end{Highlighting}
\end{Shaded}

\begin{verbatim}
##      [,1] [,2] [,3]
## [1,]    1    1    1
## [2,]    0   -1   -5
## [3,]    0    2    0
\end{verbatim}

\hypertarget{notice-that-the-pivot-in-row-two-a22-1.-this-leads-us-to-choose-the-following-elimination-matrix-e32}{%
\subsubsection{Notice that the pivot in row two, a22 = −1. This leads us
to choose the following elimination matrix
E32:}\label{notice-that-the-pivot-in-row-two-a22-1.-this-leads-us-to-choose-the-following-elimination-matrix-e32}}

\begin{Shaded}
\begin{Highlighting}[]
\NormalTok{E32 }\OtherTok{=} \FunctionTok{matrix}\NormalTok{(}\FunctionTok{c}\NormalTok{(}\DecValTok{1}\NormalTok{,}\DecValTok{0}\NormalTok{,}\DecValTok{0}\NormalTok{,}\DecValTok{0}\NormalTok{,}\DecValTok{1}\NormalTok{,}\DecValTok{2}\NormalTok{,}\DecValTok{0}\NormalTok{,}\DecValTok{0}\NormalTok{,}\DecValTok{1}\NormalTok{),}\AttributeTok{nrow=}\DecValTok{3}\NormalTok{)}
\NormalTok{E32}
\end{Highlighting}
\end{Shaded}

\begin{verbatim}
##      [,1] [,2] [,3]
## [1,]    1    0    0
## [2,]    0    1    0
## [3,]    0    2    1
\end{verbatim}

\hypertarget{this-gives-us-the-u-matrix}{%
\subsubsection{This gives us the U
matrix}\label{this-gives-us-the-u-matrix}}

\begin{Shaded}
\begin{Highlighting}[]
\NormalTok{U }\OtherTok{\textless{}{-}}\NormalTok{ E32 }\SpecialCharTok{\%*\%}\NormalTok{ E31 }\SpecialCharTok{\%*\%}\NormalTok{ E21 }\SpecialCharTok{\%*\%}\NormalTok{ A}
\NormalTok{U}
\end{Highlighting}
\end{Shaded}

\begin{verbatim}
##      [,1] [,2] [,3]
## [1,]    1    1    1
## [2,]    0   -1   -5
## [3,]    0    0  -10
\end{verbatim}

\hypertarget{we-can-compute-l-as-the-inverse-of-the-product-of-the-elimination-matrices.}{%
\subsubsection{We can compute L as the inverse of the product of the
elimination
matrices.}\label{we-can-compute-l-as-the-inverse-of-the-product-of-the-elimination-matrices.}}

\hypertarget{l-inve32-e31-e21}{%
\subsubsection{L = INV(E32 × E31 × E21)}\label{l-inve32-e31-e21}}

\begin{Shaded}
\begin{Highlighting}[]
\FunctionTok{inv}\NormalTok{(E32)}
\end{Highlighting}
\end{Shaded}

\begin{verbatim}
##      [,1] [,2] [,3]
## [1,]    1    0    0
## [2,]    0    1    0
## [3,]    0   -2    1
\end{verbatim}

\begin{Shaded}
\begin{Highlighting}[]
\FunctionTok{inv}\NormalTok{(E31)}
\end{Highlighting}
\end{Shaded}

\begin{verbatim}
##      [,1] [,2] [,3]
## [1,]    1    0    0
## [2,]    0    1    0
## [3,]    3    0    1
\end{verbatim}

\begin{Shaded}
\begin{Highlighting}[]
\FunctionTok{inv}\NormalTok{(E21)}
\end{Highlighting}
\end{Shaded}

\begin{verbatim}
##      [,1] [,2] [,3]
## [1,]    1    0    0
## [2,]    4    1    0
## [3,]    0    0    1
\end{verbatim}

\hypertarget{we-can-also-use-the-solve-function-in-r-to-get-the-inverse-of-the-matrices}{%
\subsubsection{We can also use the solve function in R to get the
inverse of the
matrices}\label{we-can-also-use-the-solve-function-in-r-to-get-the-inverse-of-the-matrices}}

\begin{Shaded}
\begin{Highlighting}[]
\NormalTok{L }\OtherTok{\textless{}{-}} \FunctionTok{solve}\NormalTok{(E21) }\SpecialCharTok{\%*\%} \FunctionTok{solve}\NormalTok{(E31) }\SpecialCharTok{\%*\%} \FunctionTok{solve}\NormalTok{(E32)}

\NormalTok{L}
\end{Highlighting}
\end{Shaded}

\begin{verbatim}
##      [,1] [,2] [,3]
## [1,]    1    0    0
## [2,]    4    1    0
## [3,]    3   -2    1
\end{verbatim}

\hypertarget{finally-we-will-solve-for-proof-that-a-lu}{%
\subsubsection{Finally, we will solve for proof that A =
LU}\label{finally-we-will-solve-for-proof-that-a-lu}}

\begin{Shaded}
\begin{Highlighting}[]
\NormalTok{L }\SpecialCharTok{\%*\%}\NormalTok{ U}
\end{Highlighting}
\end{Shaded}

\begin{verbatim}
##      [,1] [,2] [,3]
## [1,]    1    1    1
## [2,]    4    3   -1
## [3,]    3    5    3
\end{verbatim}

\hypertarget{this-proves-that-a-lu}{%
\subsubsection{This proves that A = LU}\label{this-proves-that-a-lu}}

\begin{Shaded}
\begin{Highlighting}[]
\NormalTok{(L }\SpecialCharTok{\%*\%}\NormalTok{ U }\SpecialCharTok{==}\NormalTok{ A)}
\end{Highlighting}
\end{Shaded}

\begin{verbatim}
##      [,1] [,2] [,3]
## [1,] TRUE TRUE TRUE
## [2,] TRUE TRUE TRUE
## [3,] TRUE TRUE TRUE
\end{verbatim}

\end{document}
